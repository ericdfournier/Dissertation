\begin{savequote}[75mm]
In god we trust. 
All others bring data. 
\qauthor{William Edwards Demming (1900-1993)}
\end{savequote}

\chapter{Case Study Results}
\label{Chapter 4}

\newpage

\section{Santa Barbara Region}

\newthought{The first case study region}, comprising the coastal southern portion of Santa Barbara County, was selected to reflect the local interests of the institution supporting this dissertation research. Hydrologically, this case study region is distinctly enclosed by a steep coastal mountain range to the north and the pacific ocean to the south. This case study area is not connected to any of the major inter-basin water transfer projects within the state (i.e. The State Water Project, the Los Angeles aqueduct, etc.). As such, Santa Barbara municipal water managers must be both creative and self reliant in terms of their long term municipal water supply strategies. 

Fortunately, from a freshwater management perspective, the region's unique physical geography also functions to limit the possibilities for increased population growth and urban development. Thus, the prospects for severe water shortages due to steep increases in demand are fairly unlikely. Despite this fact however, the recent drought condition throughout the state have lead to high wholesale water costs for the Santa Barbara district. This is because they are in competition with regional agricultural interests with long term investments in costly orchard based crops that cannot be left to fallow. 

In terms of alternative water supply options within the region, Santa Barbara has recently renewed talks for the development of a local seawater desalination plant that had been put on hold following the 2008 economic recession. This willingness to reconsider a high cost desalination based alternative freshwater supply strategy suggests that large scale municipal water reuse may also be put forth as a feasible alternative in the near term future and thus, that such a prospective analysis of the tradeoffs associated with such a system would indeed be valuable exercise.  

    \subsection{Regional Context}
    
    \begin{itemize}
      \setlength{\itemsep}{0cm}
      \setlength{\parskip}{0cm}
        \item HUC-8 Code: $18060013$
        \item Total Area: $1,173.6$ $km^2$
        \item Maximum Elevation: $1,376.7$ $m$
        \item Minimum Elevation: $-0.7$ $m$
        \item Mean Slope: $13.98$ $\%$
        \item Standard Deviation of Slope: $11.07$ $\%$
        \item Dominant Soil Composition: Hydrologic Soil Group - B: $10-20\%$ clay, $50-90\%$ sand, $35\%$ rock fragments
    \end{itemize}
    
        \begin{figure}[!h]
            \begin{center}
            \includegraphics[width=5.5in]{figures/SantaBarbara_Overview.png}   
            \caption{Santa Barbara Region Overview}
            \label{fig:SBoverview}
            \end{center}
        \end{figure}

    \subsection{Search Domain}

The search domain used for both the weighted overlay site suitability analysis as well as the corridor location problem specification is depicted in Figure \ref{fig:SBdomain}. The extent and dimensions of this search domain is depicted in the statistics below.
    
    \begin{itemize}
      \setlength{\itemsep}{0cm}
      \setlength{\parskip}{0cm}
        \item Grid Dimensions: $363$ $cells$ x $1351$ $cells$
        \item Grid Cell Resolution: $100$ $m$ x $100$ $m$ ($1$ $ha$)
        \item Feasible Grid Cells: $117,363$ $cells$
    \end{itemize}
    
        \begin{figure}[!h]
            \begin{center}
            \includegraphics[width=5.5in]{figures/SantaBarbara_SearchDomain.png}   
            \caption{Santa Barbara Region Search Domain}
            \label{fig:SBdomain}
            \end{center}
        \end{figure}
        
    \subsection{Destination Search Inputs}

There are three key inputs to the weighted overlay analysis used to determine the location and extent of suitable sites for the implementation of artificial groundwater recharge basins within the region. The four layers which were generated as the discrete inputs to the WOA procedure are depicted in Figures \ref{fig:SBdsinputs_slope} through \ref{fig:SBdsinputs_landuse}. The first layer gives each cell in the search domain a score between 1 and 10 on the basis of the suitability of its slope for the implementation of a artificial groundwater recharge basin. Areas with steep slopes are given lower suitability scores. Areas with shallower slopes are given high suitability scores. 

The second input to the WOA destination search process is based upon the permeability of the surface geology as shown in Figure \ref{fig:SBdsinputs_geology}. Permeability is a crucial parameter in determining the rate of infiltration that can be achieved by a recharge basin and thus the requisite size of a basin for the purpose achieving a specified total rate of recharge. The geology score layer gives each cell in the search domain a ordinal score between 1 and 10 on the basis of the underlying surface geology layer's permeability constant.

The final input to the WOA destination search process is based upon the existing landuse as shown in Figure \ref{fig:SBdsinputs_landuse}. The existing landuse can be a proxy measure of both the cost of procurement for the landholdings required to implement the artificial recharge basin as well as the regulatory and engineering difficulty associated with artificial recharge basin implementation. Here again, these scores are have been pegged to a 1 to 10 ordinal scale that aligns with those assigned to each of the other two score layers. 

        \begin{figure}[!h]
            \begin{center}
            \includegraphics[width=5.5in]{figures/SantaBarbara_Search_Slope.png}   
            \caption{Santa Barbara Region Destination Search Inputs: Slope Scores}
            \label{fig:SBdsinputs_slope}
            \end{center}
        \end{figure}
        
        \begin{figure}[!h]
            \begin{center}
            \includegraphics[width=5.5in]{figures/SantaBarbara_Search_Geology.png}   
            \caption{Santa Barbara Region Destination Search Inputs: Geology Scores}
            \label{fig:SBdsinputs_geology}
            \end{center}
        \end{figure}
    
        \begin{figure}[!h]
            \begin{center}
            \includegraphics[width=5.5in]{figures/SantaBarbara_Search_Landuse.png}   
            \caption{Santa Barbara Region Destination Search Inputs: Landuse Scores}
            \label{fig:SBdsinputs_landuse}
            \end{center}
        \end{figure}
    
    \subsection{Destination Search Outputs}
    
The raw output of the WOA destination search process is a composite layer of which depicts a measure of overall suitability for the given landuse application on an ordinal scale as in Figure \ref{fig:SBdsoutputs_comp}. This single composite suitability layer is then thresholded, selecting only those areas that have the highest composite suitability scores as shown in Figure \ref{fig:SBdsoutputs_cand}. A set of morphological operations is applied to this threshold mask which ranks each connected area of high suitability in terms of its size. Larger connected areas of high suitability are considered better in this process and thus, in this way, a single destination location for the corridor search process can be automatically selected as the single largest area of high aggregate suitability with the study area. 
    
        \begin{figure}[!h]
            \begin{center}
            \includegraphics[width=5.5in]{figures/SantaBarbara_Search_Composite.png}   
            \caption{Santa Barbara Region Destination Search Outputs: Composite Scores}
            \label{fig:SBdsoutputs_comp}
            \end{center}
        \end{figure}
        
        \begin{figure}[!h]
            \begin{center}
            \includegraphics[width=5.5in]{figures/SantaBarbara_Search_Output.png}   
            \caption{Santa Barbara Region Destination Search Outputs: Candidate Regions}
            \label{fig:SBdsoutputs_cand}
            \end{center}
        \end{figure}

    \subsection{Proposed Corridor Endpoints}
    
For the Santa Barbara case study region, the final output of the WOA analysis is shown in Figure \ref{fig:SBendpoints} in red and mapped relative to the location of the source location for the corridor location analysis that corresponds to the location of the largest WWTP within the basin, in green. These two points, plus the extent of the search domain, form the basis of the corridor location problem specification that is to be discussed in further detail in the subsequent section. 
    
    \begin{itemize}
      \setlength{\itemsep}{0cm}
      \setlength{\parskip}{0cm}
        \item Start Location: $(313,1083)$
        \item End Destination: $(248,886)$   
        \item Shortest Euclidean Path Distance: $20,745$ $m$ ($21$ $km$) 
    \end{itemize}
    
        \begin{figure}[!h]
            \begin{center}
            \includegraphics[width=5.5in]{figures/SantaBarbara_Endpoints.png}   
            \caption{Santa Barbara Region Proposed Corridor Endpoints}
            \label{fig:SBendpoints}
            \end{center}
        \end{figure}
            
    \subsection{Proposed Objective Layers}
    
For the corridor location problem specification used as the input to the MOGADOR algorithm, four key pieces of information are required. The first three correspond to the source location, the destination location, and the search domain boundaries that have been previously described. The forth key input category corresponds to the objective score layers which capture the cost associated with routing sections of a corridor over each grid cell in the search domain. For this analysis, the following three distinct objectives were developed.

The first objective category is based upon the accessibility of each location for the purposes of constructing and maintaining the water distribution infrastructure that the corridor is designed to support and is shown in Figure \ref{fig:SBaccessibility}. It is fundamentally easier to get materials and people to a locations that are positioned along road networks. As a result, the underlying road network topology was used to encode a continuous objective score layer with values ranging from 1 to 10 that can be described as a measure of "Accessibility" and which favors those locations that are on and around roads.

The second objective category is based upon the existing land use regime within the regional search domain. The idea behind the composition of this objective can be thought of as somewhat of the converse of Accessibility in the sense that, regions which are already heavily developed are likely to be socially, politically, or economically challenging to implement corridors for large scale water distribution pipeline infrastructure. Using standardized USGS based land use classification, each grid cell in the search domain is given a nominal objective score value from 1 to 10  corresponding the relative level of "Disturbance" that would be associated with routing a corridor across it. This objective layer is depicted in the layer plotted in Figure \ref{fig:SBdisturbance}.

The third objective category is derived from the underlying slope within the search domain. Steeper slopes are assigned a higher ordinal score, ranging from 1 to 10. This objective reflects the desire for corridors to be shorter in length and minimally accumulate slopes over their length. In this way, the slope score provides a mechanism for the corridor routing algorithm to preferentially favor corridors that would have minimal energy requirements in terms of the operational energy requirements of the anticipated water distribution infrastructure. This slope score objective layer is depicted in Figure \ref{fig:SBslope}. 

        \begin{figure}[!h]
            \begin{center}
            \includegraphics[width=5.5in]{figures/SantaBarbara_AccessibilityScore.png}   
            \caption{Santa Barbara Region Accessibility Based Objective Scores}
            \label{fig:SBaccessibility}
            \end{center}
        \end{figure} 

        \begin{figure}[!h]
            \begin{center}
            \includegraphics[width=5.5in]{figures/SantaBarbara_DisturbanceScore.png}   
            \caption{Santa Barbara Region Land Use Disturbance Based Objective Scores}
            \label{fig:SBdisturbance}
            \end{center}
        \end{figure}
        
        \begin{figure}[!h]
            \begin{center}
            \includegraphics[width=5.5in]{figures/SantaBarbara_SlopeScore.png}   
            \caption{Santa Barbara Region Slope Based Objective Scores}
            \label{fig:SBslope}
            \end{center}
        \end{figure}
        
    \subsection{Proposed Corridor Solutions}
    
Shown in Figure \ref{fig:SBresults} are the outputs of a series of three runs of the MOGADOR algorithm for the Santa Barbara region problem specification. These three runs differ solely in terms of the number of individuals contained within the seed population. The size of this seed population determines the extent with which the input search domain is search and, consequently, the degree to which the output solution corridor is likely to approximate the global optimal solution. The figure contains six panes made up of three rows and three columns. The columns depict, from left to right, and plan view of the final output solution set, and line plot of the break down of objective scores for the top 100 ranked individuals in the final output solution set, and, finally, a histogram plot of the frequency of total aggregate objective scores among the same top 100 individuals. Alternatively, the rows, moving from top to bottom, reflect the changing results as the population size is increased from 1,000 to 10,000 to 100,000. 

As the histogram plots of the aggregate objective scores illustrate, with a population size of 100,000 the aggregate objective scores are quite low, and the quality of the final output solution set is very high. This improvement in solution quality comes at the expense of processing time/effort. This tradeoff shall be discussed in greater detail and illustrated comparatively across all of the five case studies at the end of this chapter. 

One interesting feature of this exercise which can be readily appreciated from this set of plots is the source of the improvement in the aggregate objective scores between the different runs. For example, note the height of the data series depicted by the blue line, corresponding to the accessibility score, in the three plots in the middle column. The progressive decrease in the values associated with this line indicates that the reduction in aggregate objective scores between the three runs can be attributed to a reduction in the Accessibility score. This is tantamount to saying that the search process is able to provide better solutions as it "Finds the road network." And indeed, this conclusion is reflected from a simple visual inspection of the output corridors plotted in the panels contained in the first column. Here it can be seen that in the 100,000 population size solution set, the pathway sections have become much more linear, and appear to correspond with the layout of different road segments which occupy the area in between the source and the destination. 
    
        \begin{figure}[!h]
            \begin{center}
            \includegraphics[width=6in]{figures/SantaBarbara_PathwayResults.png}   
            \caption{Santa Barbara Region Corridor Analysis Results}
            \label{fig:SBresults}
            \end{center}
        \end{figure}
        
Figure \ref{fig:SBelevationProfile} provides an illustration of the top ranked final output corridor solution presented in the context of the full search domain. For all of the case studies the highest quality solution was developed by the model run containing the largest seed population. This was fully expected however and is in good agreement with the theoretical discussion of the role of the population initialization procedure in the behavior of the MOGADOR algorithm described in Chapter 3. l

        \begin{figure}[!h]
            \begin{center}
            \includegraphics[width=5.5in]{figures/SantaBarbara_PathwayLarge.png}   
            \caption{Santa Barbara Region Top 100 Corridors Basin Wide Overview}
            \label{fig:SBsolutionOverview}
            \end{center}
        \end{figure}
        
    \subsection{Along-Corridor Elevation Profile}
    
Figure \ref{fig:SBelevationProfile} illustrates the along-corridor elevation profile that can be generated by superimposing the output corridor solution on top of a regional digital elevation model for the Santa Barbara region. As the Figure shows the total elevation gain between the source and the destination location is a modest 200 meters across a distance spread of roughly 25 kilometers. While it may appear that the corridor has a significant amount of vertical fluctuations, these are minor in absolute terms, and stem from the fact that the slope score -- the objective most directly related to the corridor elevation profile structure -- was but only one of three in the multi-objective problem statement. These elevation fluctuations therefore can be thought of as the result of a profitable tradeoff between the accumulation of smoother slopes and more favorable values for the other two objectives. 
        
        \begin{figure}[!h]
            \begin{center}
            \includegraphics[width=5.5in]{figures/SantaBarbara_Elevation_Profile.png}   
            \caption{Santa Barbara Region Proposed Corridor Elevation Profile}
            \label{fig:SBelevationProfile}
            \end{center}
        \end{figure}
        
\clearpage
    
\section{Oxnard Region}

The second case study region consists of the HUC-8 zone containing the Oxnard plain and its immediate surrounding territories stretching as far inland as Ojai. This second case study region is located nearly adjacent to the first -- separated only by a single HUC-8 basin. The reason for the choice of two case study regions in such close physical proximity is down to the recent implementation of a functioning large scale water reuse system by the municipal water management district there. 

The majority of this HUC-8 zone's area is a comprised of a broad low lying alluvial plain. This plain has found rich application within the agricultural sector, supporting the production of a wide variety of row crops as well as high value orchard stands. Over the past three decades, the region has also experience significant population growth with sprawling suburban communities encroaching into the more marginal farmlands or those held by smaller independent farmers. The combined freshwater demands of these two sectors have conspired to create a persistent imbalance between freshwater supply and demand in this coastal region. 

Oxnard's struggle with freshwater management issues can be traced as far back as 1937 when the USGS identified that sustained groundwater pumping to support the irrigation of surface crops was contributing to the depletion of the underlying aquifer and inviting the intrusion of brackish seawater into the subsurface hydrologic strata. In response to this issue, the local municipal water authority enacted a program in which a portion of the regions' agricultural water was diverted towards a series of subsurface injection wells -- strategically positioned along the coast -- through which freshwater would be pumped to created an artificial pressure head barrier to prevent further intrusion of seawater, and thus further contamination of the aquifer. 

This program has operated successfully for a number of decades now; achieving a functional equilibrium between the level of groundwater pumping occurring within the basin and the amount of water the is delivered to the artificial intrusion barrier. More recently however, decreases in available freshwater supply due to a persistent statewide drought have forced municipal water managers in this region to thing more proactively about developing alternative sources of water supply. This process began with the creation of a plant to substitute potable freshwater for reclaimed brackish water for use in the subsurface barrier injection wells. 

The successful operation of this plant for a number of years inspired enough confidence among the water resource management authorities in this area to pursue and very recently achieve a goal of implementing a facility capable of reclaiming and reusing the growing volume of municipal wastewater being generated within the basin. This new facility, commissioned just this year, provides the capability to treat 100\% of the wastewater generated in the basin to a potable standard through a complex treatment chain incorporating a sophisticated chain of tertiary treatment processes including: advanced micro-filtration, reverse osmosis, ultraviolet filtration, and ozonation. The long term plan for the water currently being produced by this facility is for groundwater recharge at higher elevation locations within the basin. As such, this locale represents the ideal candidate for evaluation in this study. 

    \subsection{Regional Context}
    
    \begin{itemize}
      \setlength{\itemsep}{0cm}
      \setlength{\parskip}{0cm}
        \item HUC-8 Code: $18070102$
        \item Total Area: $5,188.3$ $km^2$
        \item Maximum Elevation: $2,664.4$ $m$
        \item Minimum Elevation: $-0.05$ $m$
        \item Mean Slope: $15.54$ $\%$
        \item Standard Deviation of Slope: $11.11$ $\%$
        \item Dominant Soil Composition: Hydrologic Soil Group - B: $10-20\%$ clay, $50-90\%$ sand, $35\%$ rock fragments
    \end{itemize}
    
        \begin{figure}[!h]
            \begin{center}
            \includegraphics[width=5.5in]{figures/Oxnard_Overview.png}   
            \caption{Oxnard Region Overview}
            \label{fig:Ooverview}
            \end{center}
        \end{figure}

    \subsection{Search Domain}
    
The search domain comprising the Oxnard study region is described in the statistics below and depicted graphically in the map panel contained within \ref{fig:Odomain}. Relative to the total land area contained within the Santa Barbara study region, the Oxnard domain is quite large, being nearly four times its total size. 
    
    \begin{itemize}
      \setlength{\itemsep}{0cm}
      \setlength{\parskip}{0cm}
        \item Grid Dimensions: $677$ $cells$ x $1586$ $cells$
        \item Grid Cell Resolution: $100$ $m$ x $100$ $m$ ($1$ $ha$)
        \item Feasible Grid Cells: $518,834$ $cells$
    \end{itemize}
    
        \begin{figure}[!h]
            \begin{center}
            \includegraphics[width=5.5in]{figures/Oxnard_SearchDomain.png}   
            \caption{Oxnard Region Search Domain}
            \label{fig:Odomain}
            \end{center}
        \end{figure}
        
    \subsection{Destination Search Inputs}
    
In Figures \ref{fig:Odsinputs_slope} through \ref{fig:Odsinputs_landuse} the three key inputs to the Oxnard reuse destination search process are shown. A visual inspection of these three layers reveals that there is an obvious band of continuously high suitability stretching from the foot of the basin (at the lower left) along its lower portion nearly across its breadth (to the lower right). This corridor is flat low lying river bed. It possesses a highly permeable surface geology, a very shallow slope profile, and relatively low intensity land use applications. 
    
        \begin{figure}[!h]
            \begin{center}
            \includegraphics[width=5.5in]{figures/Oxnard_Search_Slope.png}   
            \caption{Oxnard Region Destination Search Inputs: Slope Scores}
            \label{fig:Odsinputs_slope}
            \end{center}
        \end{figure}
        
        \begin{figure}[!h]
            \begin{center}
            \includegraphics[width=5.5in]{figures/Oxnard_Search_Geology.png}   
            \caption{Oxnard Region Destination Search Inputs: Geology Scores}
            \label{fig:Odsinputs_geology}
            \end{center}
        \end{figure}
    
        \begin{figure}[!h]
            \begin{center}
            \includegraphics[width=5.5in]{figures/Oxnard_Search_Landuse.png}   
            \caption{Oxnard Region Destination Search Inputs: Landuse Scores}
            \label{fig:Odsinputs_landuse}
            \end{center}
        \end{figure}
    
    \subsection{Destination Search Outputs}
    
The layer showing the composite suitability of each cell within the study site for the site of a destination artificial water reuse facility is shown in Figure \ref{fig:Odsoutputs_comp}. As this figure shows, the area with the highest composite suitability is that which was mentioned previously as being clearly visible within each of the individual input suitability layers. The majority of the top ranked areas of contiguous high suitability are contained within this region, as shown in Figure \ref{fig:Odsoutputs_cand}.
    
        \begin{figure}[!h]
            \begin{center}
            \includegraphics[width=5.5in]{figures/Oxnard_Search_Composite.png}   
            \caption{Oxnard Region Destination Search Outputs: Composite Scores}
            \label{fig:Odsoutputs_comp}
            \end{center}
        \end{figure}
        
        \begin{figure}[!h]
            \begin{center}
            \includegraphics[width=5.5in]{figures/Oxnard_Search_Output.png}   
            \caption{Oxnard Region Destination Search Outputs: Candidate Regions}
            \label{fig:Odsoutputs_cand}
            \end{center}
        \end{figure}

    \subsection{Proposed Corridor Endpoints}
    
    For the Oxnard case study region, the final output of the WOA analysis is shown in Figure \ref{fig:Oendpoints} in red and mapped relative to the location of the source location for the corridor location analysis that corresponds to the location of the largest WWTP within the basin, in green. These two points, plus the extent of the search domain, form the basis of the corridor location problem specification that is to be discussed in further detail in the subsequent section. 
    
    \begin{itemize}
      \setlength{\itemsep}{0cm}
      \setlength{\parskip}{0cm}
        \item Start Location: $(656,236)$
        \item End Destination: $(513,532)$
        \item Shortest Euclidean Path Distance: $32,873$ $m$ ($32$ $km$)
    \end{itemize}
    
        \begin{figure}[!h]
            \begin{center}
            \includegraphics[width=5.5in]{figures/Oxnard_Endpoints.png}   
            \caption{Oxnard Region Proposed Corridor Endpoints}
            \label{fig:Oendpoints}
            \end{center}
        \end{figure}
            
    \subsection{Proposed Objective Layers}
    
In Figures \ref{fig:Oaccessibility} through \ref{fig:Oslope} the three independent objective layers used as inputs to the MOGADOR problem specification for the Oxnard study site are shown. These three layers correspond to the categories of landuse disturbance, accessibility, and slope described previously for the Santa Barbara case study region and used for all of the other case studies in the analysis. 

        \begin{figure}[!h]
            \begin{center}
            \includegraphics[width=5.5in]{figures/Oxnard_AccessibilityScore.png}   
            \caption{Oxnard Region Accessibility Based Objective Scores}
            \label{fig:Oaccessibilty}
            \end{center}
        \end{figure}

        \begin{figure}[!h]
            \begin{center}
            \includegraphics[width=5.5in]{figures/Oxnard_DisturbanceScore.png}   
            \caption{Oxnard Region Land Use Based Disturbance Objective Scores}
            \label{fig:Odisturbance}
            \end{center}
        \end{figure}
        
        \begin{figure}[!h]
            \begin{center}
            \includegraphics[width=5.5in]{figures/Oxnard_SlopeScore.png}   
            \caption{Oxnard Region Slope Based Objective Scores}
            \label{fig:Oslope}
            \end{center}
        \end{figure}
        
    \subsection{Proposed Corridor Solutions}

Figure \ref{fig:Oresults} presents a figure panel containing the outputs of the three separate MOGADOR algorithm runs for the Oxnard study site problem specification using three different population sizes. As this figure panel illustrates, the first algorithm run, with a population size of 1,000, delivered a set of 100 top output corridor solutions with aggregate objective score values ranging from 3585 to 3615. With the second run of the algorithm, where the population size was increased to a 10,000, the top 100 output corridor solutions' aggregate objective scores can be observed to have improved markedly, covering a range from 2910 to 2935. The line plot in the center of the figure attests to the fact that this improvement came from reductions in both the accessibility and disturbance scores associated with the new output corridor set.

The final corridor solution set, generated from a MOGADOR model run where the input population size was fixed at a value of 100,000, produced an output set of top 100 solutions with an even lower range of composite objective scores: ranging from between 2740 to 2760. The improvement in this overall composite objective scores, as shown by the line plot at the lower center portion of the figure panel, can be observed as being attributable to marginal reductions in the accessibility and disturbance objective scores.  This can be interpreted as the algorithm locating corridors which route around areas with high intensity land uses and routing along the more highly accessible transportation network.

Figure \ref{fig:OsolutionOverview} provides a broad plan overview illustration of the final output corridors produced by the MOGADOR solution run where the population size was set to a value of 100,000. 


        \begin{figure}[!h]
            \begin{center}
            \includegraphics[width=6in]{figures/Oxnard_PathwayResults.png}   
            \caption{Oxnard Region Corridor Analysis Results}
            \label{fig:Oresults}
            \end{center}
        \end{figure}

        \begin{figure}[!h]
            \begin{center}
            \includegraphics[width=5.5in]{figures/Oxnard_PathwayLarge.png}   
            \caption{Oxnard Region Top 100 Corridors Basin Wide Overview}
            \label{fig:OsolutionOverview}
            \end{center}
        \end{figure}
        
The corresponding elevation profile for the solution illustrated in Figure \ref{fig:OsolutionOverview} can be observed plotted in Figure \ref{fig:OelevationProfile}. This elevation profile reals that there is only a modest amount of elevation change along the length of the proposed corridor solution (roughly 130 meters). It also shows that this elevation gain is discontinuous along the length of the proposed corridor solution with there being to two modest hills -- each roughly 40 meters in height -- that must be ascended and descended before the destination is reached. 
        
        \begin{figure}[!h]
            \begin{center}
            \includegraphics[width=5.5in]{figures/Oxnard_Elevation_Profile.png}
            \caption{Oxnard Region Proposed Corridor Elevation Profile}
            \label{fig:OelevationProfile}
            \end{center}
        \end{figure}   
    
\clearpage    
    
\section{San Diego Region}

The third case study region investigated as part of this dissertation consists of the HUC-8 basin comprising the city of San Diego and its adjacent metropolitan districts. This basin is situated at the southwestern most tip of the State of California and adjoins the international border between the United States and Mexico as shown by the black filled area plotted in Figure \ref{fig:SDoverview}.

In terms of its water resource management history, San Diego has become world renowned as a leader in freshwater management for its innovative approaches to demand management policy and advanced technological solutions for providing alternatives supply. As is often the case, necessity has been the mother of this innovation, with the San Diego region experiencing explosive growth in population growth and associated freshwater demand over the past 50 years while at the same time being cutoff from major statewide inter-basin water transfer projects.

For example, from 2009 to 2013 the San Diego Municipal Water District embarked upon a large scale demonstration project to determine whether the advanced tertiary water treatment systems that would be necessary to facilitate large scale indirect, or possibly even direct, potable reuse could be implemented effectively and reliably at scale. In this project, purified water was blended with imported water supplies in the San Vicente Reservoir before going to the standard drinking water treatment plant. Due in large part to this success of the pilot program, the San Diego city council recently unanimously approved a three and a half billion dollar direct potable reuse project and plant that is to be constructed over the next decade. This facility is being planned in conjunction with another large scale desalination plant in a bid to build a portfolio of alternative freshwater supply and groundwater recharge capacity just as the state enters the fourth year of a crippling drought condition.

The San Diego case study region is unique in the context of the other case study regions evaluated as part of this dissertation in that the destination location to be used for the corridor location problem specification has been designated as the location of the municipal drinking water treatment plant. As such, no destination search process was undertaken for this case study region.

    \subsection{Regional Context}
    
    \begin{itemize}
      \setlength{\itemsep}{0cm}
      \setlength{\parskip}{0cm}
        \item HUC-8 Code: $18070304$
        \item Total Area: $4,338.1$ $km^2$
        \item Maximum Elevation: $1,977$ $m$
        \item Minimum Elevation: $-0.7$ $m$
        \item Mean Slope: $9.38$ $\%$
        \item Standard Deviation of Slope: $8.77$ $\%$
        \item Dominant Soil Composition: Hydrologic Soil Group - B: $10-20\%$ clay, $50-90\%$ sand, $35\%$ rock fragments
    \end{itemize}
    
        \begin{figure}[!h]
            \begin{center}
            \includegraphics[width=5.5in]{figures/SanDiego_Overview.png}   
            \caption{San Diego Region Overview}
            \label{fig:SDoverview}
            \end{center}
        \end{figure}

    \subsection{Search Domain}
    
    The search domain comprising the San Diego study region is described in the statistics below and depicted graphically in the map panel contained within \ref{fig:SDdomain}.
        
    \begin{itemize}
      \setlength{\itemsep}{0cm}
      \setlength{\parskip}{0cm}
        \item Grid Dimensions: $798$ $cells$ x $898$ $cells$
        \item Grid Cell Resolution: $100$ $m$ x $100$ $m$ ($1$ $ha$)
        \item Feasible Grid Cells: $433,808$ $cells$
    \end{itemize}
    
        \begin{figure}[!h]
            \begin{center}
            \includegraphics[width=5.5in]{figures/SanDiego_SearchDomain.png}   
            \caption{San Diego Region Search Domain}
            \label{fig:SDdomain}
            \end{center}
        \end{figure}

    \subsection{Proposed Corridor Endpoints}
    
The proposed endpoints to be used in the MOGADOR algorithm specification are shown in Figure \ref{fig:SDendpoints}. The source location was determined by the location of the largest NPDES permitted WWTP in the basin while the destination location, in this case, was pre-determined as the location at which an artificial groundwater recharge basin has already been implemented.
    
    \begin{itemize}
      \setlength{\itemsep}{0cm}
      \setlength{\parskip}{0cm}
        \item Start Location: $(635,42)$
        \item End Destination: $(453,363)$    
        \item Shortest Euclidean Path Distance: $36,901$ $m$ ($36$ $km$)
    \end{itemize}
    
        \begin{figure}[!h]
            \begin{center}
            \includegraphics[width=5.5in]{figures/SanDiego_Endpoints.png}   
            \caption{San Diego Region Proposed Corridor Endpoints}
            \label{fig:SDendpoints}
            \end{center}
        \end{figure}

    \subsection{Proposed Objective Layers}
    
The three proposed objective layers which round out the MOGADOR algorithm problem specification and depicted in Figures \ref{fig:SDdisturbance} through \ref{fig:SDslope}, consist of the same accessibility, landuse disturbance, and slope based data layers as those described previously for Santa Barbara \& Oxnard, and used for all of the case studies included in this analysis. In terms of the structure of these objectives in the San Diego region, the coastal areas tend to be very highly developed with large land use disturbance scores as well as a dense road network providing favorable accessibility values. The basin does not contain an extreme amount of topographic relief as evidenced by the fairly homogeneous distribution of slopes shown in Figure \ref{fig:SDslope}.

        \begin{figure}[!h]
            \begin{center}
            \includegraphics[width=5.5in]{figures/SanDiego_AccessibilityScore.png}   
            \caption{San Diego Region Accessibility Based Objective Scores}
            \label{fig:SDaccessibilty}
            \end{center}
        \end{figure}

        \begin{figure}[!h]
            \begin{center}
            \includegraphics[width=5.5in]{figures/SanDiego_DisturbanceScore.png}   
            \caption{San Diego Region Land Use Disturbance Based Objective Scores}
            \label{fig:SDdisturbance}
            \end{center}
        \end{figure}
        
        \begin{figure}[!h]
            \begin{center}
            \includegraphics[width=5.5in]{figures/SanDiego_SlopeScore.png}   
            \caption{San Diego Region Slope Based Objective Scores}
            \label{fig:SDslope}
            \end{center}
        \end{figure}
        
    \subsection{Proposed Corridor Solutions}

The results of the three runs of the MOGADOR algorithm for the San Diego region case study corridor location analysis are presented in Figure \ref{SDresults} and reflect the same three variations on the seed population size described for the previous case studies. Here again, the highest quality solution set was produced by the MOGADOR run using the largest population size with the minimum cumulative objective scores for the top 100 output solutions ranging from 3840 to 3865. Between the three algorithm runs, the majority of the aggregate objective score improvement came from reductions in the accessibility and disturbance scores; this, again, reflecting the algorithm's iterative discovery of those corridor sections running in and along road network sections while avoiding areas with higher intensity landuse.

The top output solution for the three runs is depicted in the context of the entire search domain in Figure \ref{fig:SDsolutionOverview}. As this Figure shows the final corridor routes from the location of the coastal WWTP treatment plant, around the large harbor area at the Southwest portion of the search domain, and up towards the existing groundwater recharge facility located at the heart of the basin. 
    
        \begin{figure}[!h]
            \begin{center}
            \includegraphics[width=6in]{figures/SanDiego_PathwayResults.png}   
            \caption{San Diego Region Corridor Analysis Results}
            \label{fig:SDresults}
            \end{center}
        \end{figure}

        \begin{figure}[!h]
            \begin{center}
            \includegraphics[width=5.5in]{figures/SanDiego_PathwayLarge.png}   
            \caption{San Diego Region Top 100 Corridors Basin Wide Overview}
            \label{fig:SDsolutionOverview}
            \end{center}
        \end{figure}
        
Figure \ref{fig:SDelevationProfile} plots the elevation profile of the land surface along the length of the proposed corridor solution. The maximum elevation gain between the endpoints of the corridor is roughly 230 meters, however there is a considerable amount of ups and downs along the corridor's length. The jaggedness of the elevation profile reflects the extremely high density of the urban environment in the area immediately inland from the coastal WWTP. It reflects a necessary tradeoff between the accumulation of slope and the need to route around areas with high intensity existing landuse. 
        
        \begin{figure}[!h]
            \begin{center}
            \includegraphics[width=5.5in]{figures/SanDiego_Elevation_Profile.png}   
            \caption{Santa Diego Region Proposed Corridor Elevation Profile}
            \label{fig:SDelevationProfile}
            \end{center}
        \end{figure}

\clearpage

\section{Santa Ana -- San Bernadino Region}

The Santa Ana -- San Bernadino case study region, filled in black in Figure \ref{fig:SASBoverview}, is comprised of another coastal HUC-8 basin that is situated north of San Diego and south of Oxnard. The majority of the basin's area is positioned inland with a small strip of land stretching westward towards the coast around the bed of the Santa Ana river. In terms of total area, the Santa Ana -- San Bernadino is the second largest case study region being investigated as part of this dissertation analysis. It is almost four times the size of the Santa Barbara case study basin and is only marginally smaller in size that the largest of the five study sites: Fresno -- Tulare.

The Santa Ana -- San Bernadino region, which is positioned squarely within Orange County, shares a number of hydrologic similarities to the San Diego region and as a result, it too has been forced to adopt a whole suite of innovative water resource management policies and technological solutions. In fact, this region is home to the first large scale commercial municipal wastewater recycling and reuse installation in the United States; called Water Factory 21. This facility takes raw sewage as influent as used a cutting edge treatment process chain to return that water to levels of purity that are of near potable standard. This reclaimed water is then pumped uphill to a series of interconnected recharge basins positioned along the bed of the Santa Ana river where it is allowed to infiltrate back into the subsurface aquifer, providing a crucial source of artificial recharge.

The Santa Ana -- San Bernadino case study site provides a unique opportunity to benchmark the results of this modeling framework against an existing reuse facility that incorporates a significant component of artificial groundwater recharge. In this way, we can do things like compare the layout of this corridor solution proposed for this region to that implemented in the real world, as well as, hopefully in the future, evaluate the estimates for the water-energy usage efficiency associated with the proposed systems specification to that experienced by the Water Factory 21 facility. 

    \subsection{Regional Context}

    \begin{itemize}
      \setlength{\itemsep}{0cm}
      \setlength{\parskip}{0cm}
        \item HUC-8 Code: $18070203$
        \item Total Area: $5,375.9$ $km^2$
        \item Maximum Elevation: $3,461.3$ $m$
        \item Minimum Elevation: $-0.7$ $m$
        \item Mean Slope: $10.56$ $\%$
        \item Standard Deviation of Slope: $12.21$ $\%$
        \item Dominant Soil Composition: Hydrologic Soil Group - B: $10-20\%$ clay, $50-90\%$ sand, $35\%$ rock fragments
    \end{itemize}
    
        \begin{figure}[!h]
            \begin{center}
            \includegraphics[width=5.5in]{figures/SanBernadino_Overview.png}   
            \caption{Santa Ana -- San Bernadino Region Overview}
            \label{fig:SASBoverview}
            \end{center}
        \end{figure}

    \subsection{Search Domain}

    The search domain comprising the San Diego study region is described in the statistics below and depicted graphically in the map panel contained within \ref{fig:SASBdomain}.
    
    \begin{itemize}
      \setlength{\itemsep}{0cm}
      \setlength{\parskip}{0cm}
        \item Grid Dimensions: $854$ $cells$ x $1463$ $cells$
        \item Grid Cell Resolution: $100$ $m$ x $100$ $m$ ($1$ $ha$)
        \item Feasible Grid Cells: $537,587$ $cells$
    \end{itemize}
    
        \begin{figure}[!h]
            \begin{center}
            \includegraphics[width=5.5in]{figures/SanBernadino_SearchDomain.png}   
            \caption{Santa Ana -- San Bernadino Region Search Domain}
            \label{fig:SASBdomain}
            \end{center}
        \end{figure}
        
\subsection{Destination Search Inputs}

In Figures \ref{fig:SASBdsinputs_slope} through \ref{fig:SASBdsinputs_landuse} the three key inputs to the Santa Ana -- San Bernadino region case study reuse destination search process are shown. Here again, a visual inspection of these three layers reveals that there is a large central plain of highly suitable areas in the left central portion of the basin. This area of high suitability is connected to the coast region, where the WWTP is located by a thin strip of land area that is marginally suitable according to the three separated suitability layers which runs along the bed of the Santa Ana river. 
    
        \begin{figure}[!h]
            \begin{center}
            \includegraphics[width=5.5in]{figures/SanBernadino_Search_Slope.png}   
            \caption{Santa Ana -- San Bernadino Region Destination Search Inputs: Slope Scores}
            \label{fig:SASBdsinputs_slope}
            \end{center}
        \end{figure}
        
        \begin{figure}[!h]
            \begin{center}
            \includegraphics[width=5.5in]{figures/SanBernadino_Search_Geology.png}   
            \caption{Santa Ana -- San Bernadino Region Destination Search Inputs: Geology Scores}
            \label{fig:SASBdsinputs_geology}
            \end{center}
        \end{figure}
    
        \begin{figure}[!h]
            \begin{center}
            \includegraphics[width=5.5in]{figures/SanBernadino_Search_Landuse.png}   
            \caption{Santa Ana -- San Bernadino Region Destination Search Inputs: Landuse Scores}
            \label{fig:SASBdsinputs_landuse}
            \end{center}
        \end{figure}
    
    \subsection{Destination Search Outputs}
    
The output of the weighted overlay analysis used to engage in the search for suitable sites for the application of the artificial groundwater recharge surface infiltration basin are shown in the composite site suitability layer depicted in Figure \ref{fig:SASBoutputs_comp}. The largest patches of contiguous high suitability are highlighted in the red portions of Figure \ref{SASBoutputs_cand}. The obvious best candidate for a recharge basin site within this search domain can been seen as the large red area positioned along the left center edge of the basin.
    
        \begin{figure}[!h]
            \begin{center}
            \includegraphics[width=5.5in]{figures/SanBernadino_Search_Composite.png}   
            \caption{Santa Ana -- San Bernadino Region Destination Search Outputs: Composite Scores}
            \label{fig:SASBdsoutputs_comp}
            \end{center}
        \end{figure}
        
        \begin{figure}[!h]
            \begin{center}
            \includegraphics[width=5.5in]{figures/SanBernadino_Search_Output.png}   
            \caption{Santa Ana -- San Bernadino Region Destination Search Outputs: Candidate Regions}
            \label{fig:SASBdsoutputs_cand}
            \end{center}
        \end{figure}

    \subsection{Proposed Corridor Endpoints}
    
The proposed endpoints for the Santa Ana -- San Bernadino corridor location problem specification to be delivered to the MOGADOR algorithm are plotted in Figure \ref{fig:SASBendpoints}. The location of the destination site has been selected as the centroid of the large contiguous area of high suitability referenced in the previous section.
    
    \begin{itemize}
      \setlength{\itemsep}{0cm}
      \setlength{\parskip}{0cm}
        \item Start Location: $(840,48)$
        \item End Destination: $(528,430)$
        \item Shortest Euclidean Path Distance: $49,322$ $m$ ($49$ $km$)
    \end{itemize}
    
        \begin{figure}[!h]
            \begin{center}
            \includegraphics[width=5.5in]{figures/SanBernadino_Endpoints.png}   
            \caption{Santa Ana -- San Bernadino Region Proposed Corridor Endpoints}
            \label{fig:SASBendpoints}
            \end{center}
        \end{figure}
    
    \subsection{Proposed Objective Layers}
    
The three proposed objective layers to be used as part of the MOGADOR algorithm corridor location problem specification are illustrated graphically in Figures \ref{fig:SASBaccessibility} through \ref{fig:SASBdisturbance}. Structurally, these objective layers were generated according to the same procedures used to generate the corresponding objective layers for each one of the other five case study sites. As the figures show, the coastal area is a flat, low lying spit with a high average landuse intensity and a very dense road network. This coastal region is largely separated from other populated areas in the basin's interior by coastal mountain range with only a narrow passage having been cut by the Santa Ana River.  

        \begin{figure}[!h]
            \begin{center}
            \includegraphics[width=5.5in]{figures/SanBernadino_AccessibilityScore.png}   
            \caption{Santa Ana -- San Bernadino Region Accessibility Based Objective Scores}
            \label{fig:SASBaccessibilty}
            \end{center}
        \end{figure}

        \begin{figure}[!h]
            \begin{center}
            \includegraphics[width=5.5in]{figures/SanBernadino_DisturbanceScore.png}   
            \caption{Santa Ana -- San Bernadino Region Land Use Disturbance Based Objective Scores}
            \label{fig:SASBdisturbance}
            \end{center}
        \end{figure}
        
        \begin{figure}[!h]
            \begin{center}
            \includegraphics[width=5.5in]{figures/SanBernadino_SlopeScore.png}   
            \caption{Santa Ana -- San Bernadino Region Slope Based Objective Scores}
            \label{fig:SASBslope}
            \end{center}
        \end{figure}
        
    \subsection{Proposed Corridor Solutions}
    
As illustrated in Figure \ref{fig:SASBresults} The proposed corridor solutions for the three MOGADOR model runs with initial population sizes of 1,000, 10,000, and 100,000 show some interesting results. For example, with a population size of 1,000, the algorithm is not able to explore enough of the decision space to produce an output solution set which does not exit the search domain boundary. As a result of this, the range of cumulative aggregate objective function values for the top 100 solutions produced by this run of the algorithm are enormous in size, reflecting the arbitrarily high objective scores assigned to all grid cells outside the feasible search domain for all of the objectives. 

As the population size is increased however, we can see that the output solution sets begin to respect the search domain boundary and that the composite aggregate objective scores for the top 100 solutions in each set decrease dramatically. The top 100 solutions for the MOGADOR run with a population size of 100,000 can be seen to have relatively lower disturbance scores compared to the top 100 solutions generated by the algorithm run with a population size of 10,000. This difference reflects that ability of the run with the larger population size to route corridors that minimally disturb areas within the study site with intensive or otherwise sensitive existing landuse types.

The top 100 solutions generated by the MOGADOR algorithm run with a population of 100,000 are plotted relative to the entire study site's search domain in Figure \ref{fig:SASBsolutionOverview}.
    
        \begin{figure}[!h]
            \begin{center}
            \includegraphics[width=6in]{figures/SanBernadino_PathwayResults.png}   
            \caption{Santa Ana -- San Bernadino Region Corridor Analysis Results}
            \label{fig:SASBresults}
            \end{center}
        \end{figure}

        \begin{figure}[!h]
            \begin{center}
            \includegraphics[width=5.5in]{figures/SanBernadino_PathwayLarge.png}   
            \caption{Santa Ana -- San Bernadino Region Top 100 Corridors Basin Wide Overview}
            \label{fig:SASBsolutionOverview}
            \end{center}
        \end{figure}
        
Figure \ref{fig:SASBelevationProfile} plots the along corridor elevation profile for the best corridor solution produced as an output of the MOGADOR algorithm. This elevation profile reveals that the Santa Ana -- San Bernadino study site has the largest net elevation gradient of all of the case study regions: a total of 350 meters. It also shows that the accumulation of elevation along the length of the corridor is fairly continuous with two small declines which must be navigated at the very end section of the proposed corridor. 
        
        \begin{figure}[!h]
            \begin{center}
            \includegraphics[width=5.5in]{figures/SanBernadino_Elevation_Profile.png}   
            \caption{San Bernadino Region Proposed Corridor Elevation Profile}
            \label{fig:SASBelevationProfile}
            \end{center}
        \end{figure}

\clearpage

\section{Fresno -- Tulare Region}

The fifth and final case study region is comprised of the HUC-8 basin containing the cities of Fresno and Tulare. This case study region is depicted graphically by the region filled in black in Figure \ref{fig:Foverview}. This basin differs from those included in the previous case studies in that it is internally drained -- i.e. landlocked. This unique hydrologic feature is due to the fact that it is situated within California's Central Valley. In terms of total area, Fresno is the largest of the five case study regions. However, its topography and landuse characteristics are significantly more homogeneous than those for the other case study regions. Fresno -- Tulare, and indeed the majority of the southern portion of the California Central Valley are heavily agricultural. This is a result of the area's high quality soils, evenly flat topography, and favorable climactic regime with a large number of annual growing days. 

Due to the historic prominence of agricultural activity in this region, the local economy of the Fresno -- Tulare region is heavily skewed towards agricultural activity. As a result, there has evolved a sort of lock-in effect where the region is relied upon to produce crop outputs for national and international export regardless of the local precipitation patterns. For decades now, this situation has caused farmers to turn to local groundwater resources to offset deficiencies in freshwater supply in periods of drought. This has lead to unsustainable rates of subsurface aquifer withdrawals; on state of groundwater overdraft that in many places within the study site threaten the long term health and viability of the aquifers. 

In an effort to fight harmful consequences of this overdraft condition such as groundwater contamination and land surface subsidence, regional freshwater managers have been exploring the use of treated wastewater to provide and artificial source of groundwater recharge. The need to better understand the energy-water usage efficiencies of these types of reuse systems in this area make it a prime candidate for assessment as part of this dissertation project.  

    \subsection{Regional Context}

    \begin{itemize}
      \setlength{\itemsep}{0cm}
      \setlength{\parskip}{0cm}
        \item HUC-8 Code: $18030009$
        \item Total Area: $6,943.6$ $km^2$
        \item Maximum Elevation: $1,536.6$ $m$
        \item Minimum Elevation: $0$ $m$
        \item Mean Slope: $2.16$ $\%$
        \item Standard Deviation of Slope: $6.24$ $\%$
        \item Dominant Soil Composition: Hydrologic Soil Group - B: $10-20\%$ clay, $50-90\%$ sand, $35\%$ rock fragments
    \end{itemize}

        \begin{figure}[!h]
            \begin{center}
            \includegraphics[width=5.5in]{figures/Fresno_Overview.png}   
            \caption{Fresno -- Tulare Region Overview}
            \label{fig:Foverview}
            \end{center}
        \end{figure}

    \subsection{Search Domain}

 The search domain comprising the Fresno -- Tulare case study region is described in the statistics below and depicted graphically in the map panel contained within \ref{fig:Fdomain}.
    
    \begin{itemize}
      \setlength{\itemsep}{0cm}
      \setlength{\parskip}{0cm}
        \item Grid Dimensions: $1018$ $cells$ x $1459$ $cells$
        \item Grid Cell Resolution: $100$ $m$ x $100$ $m$ ($1$ $ha$)
        \item Feasible Grid Cells: $694,365$ $cells$
    \end{itemize}
    
        \begin{figure}[!h]
            \begin{center}
            \includegraphics[width=5.5in]{figures/Fresno_SearchDomain.png}   
            \caption{Fresno -- Tulare Region Search Domain}
            \label{fig:Fdomain}
            \end{center}
        \end{figure}
        
    \subsection{Destination Search Inputs}
    
The destination search inputs for the siting of the groundwater recharge basin are illustrated in Figures \ref{fig:Fdsinputs_slope} through \ref{fig:Fdsinputs_landuse}. Here again, as with the four other case study regions, these are comprised of the same slope, surface geology, and landuse based nominally standardized spatial data layers.
    
        \begin{figure}[!h]
            \begin{center}
            \includegraphics[width=5.5in]{figures/Fresno_Search_Slope.png}   
            \caption{Fresno -- Tulare Region Destination Search Inputs: Slope Scores}
            \label{fig:Fdsinputs_slope}
            \end{center}
        \end{figure}
        
        \begin{figure}[!h]
            \begin{center}
            \includegraphics[width=5.5in]{figures/Fresno_Search_Geology.png}   
            \caption{Fresno -- Tulare Region Destination Search Inputs: Geology Scores}
            \label{fig:Fdsinputs_geology}
            \end{center}
        \end{figure}
    
        \begin{figure}[!h]
            \begin{center}
            \includegraphics[width=5.5in]{figures/Fresno_Search_Landuse.png}   
            \caption{Fresno -- Tulare Region Destination Search Inputs: Landuse Scores}
            \label{fig:Fdsinputs_landuse}
            \end{center}
        \end{figure}
    
    \subsection{Destination Search Outputs}
    
In absolute terms, the majority of the area contained within the Fresno -- Tulare search domain is actually quite highly suitable for the implementation of a groundwater recharge basin. In terms of relative suitability however, the most favorable region was found to be located in the Northwestern portion of the search domain adjacent to an existing surface water feature and depicted graphically by the connected candidate regions plotted in Figure \ref{fig:Fdsoutputs_cand}. 
    
        \begin{figure}[!h]
            \begin{center}
            \includegraphics[width=5.5in]{figures/Fresno_Search_Composite.png}   
            \caption{Fresno -- Tulare Region Destination Search Outputs: Composite Scores}
            \label{fig:Fdsoutputs_comp}
            \end{center}
        \end{figure}
        
        \begin{figure}[!h]
            \begin{center}
            \includegraphics[width=5.5in]{figures/Fresno_Search_Output.png}   
            \caption{Fresno -- Tulare Region Destination Search Outputs: Candidate Regions}
            \label{fig:Fdsoutputs_cand}
            \end{center}
        \end{figure}

    \subsection{Proposed Corridor Endpoints}
    
The proposed corridor endpoints for the MOGADOR algorithm problem specification are plotted in Figure \ref{fig:Fendpoints}. The spacing between the source and destination cells within the search domain is the largest of that for any of the case studies included  in this entire analysis. This feature, combined with the sheer size of the Fresno -- Tulare regions search domain makes it a challenging candidate candidate for the corridor location optimization procedure. 
    
    \begin{itemize}
      \setlength{\itemsep}{0cm}
      \setlength{\parskip}{0cm}
        \item Start Location: $(435,1037)$
        \item End Destination: $(421,387)$
        \item Shortest Euclidean Path Distance: $65,015$ $m$ ($65$ $km$)
    \end{itemize}
    
        \begin{figure}[!h]
            \begin{center}
            \includegraphics[width=5.5in]{figures/Fresno_Endpoints.png}   
            \caption{Fresno -- Tulare Region Proposed Corridor Endpoints}
            \label{fig:Fendpoints}
            \end{center}
        \end{figure}

    \subsection{Proposed Objective Layers}
    
The proposed objective for the MOGADOR algorithm problem specification are plotted graphically in Figures \ref{fig:Faccessibility} through \ref{fig:Fslope}. More so than for any other case study region investigated as part of this analysis, these layers are homogeneous in structure. This presents an additional challenge to the operation of the corridor location optimization procedure. 
    
        \begin{figure}[!h]
            \begin{center}
            \includegraphics[width=5.5in]{figures/Fresno_AccessibilityScore.png}   
            \caption{Fresno -- Tulare Region Accessibility Based Objective Scores}
            \label{fig:Faccessibility}
            \end{center}
        \end{figure}

        \begin{figure}[!h]
            \begin{center}
            \includegraphics[width=5.5in]{figures/Fresno_DisturbanceScore.png}   
            \caption{Fresno -- Tulare Region Land Use Disturbance Based Objective Scores}
            \label{fig:Fdisturbance}
            \end{center}
        \end{figure}
        
        \begin{figure}[!h]
            \begin{center}
            \includegraphics[width=5.5in]{figures/Fresno_SlopeScore.png}   
            \caption{Fresno -- Tulare Region Slope Based Objective Scores}
            \label{fig:Fslope}
            \end{center}
        \end{figure}
        
    \subsection{Proposed Corridor Solutions}
    
The characteristics of the proposed corridor location solutions generated as outputs from the MOGADOR algorithm for three different population sizes are depicted in Figure \ref{fig:Fresults}. As this figure shows, and as might be expected, the best final solution set by the algorithm very closely approximates the euclidean shortest path between the input source and the input destination provided in the problem specification. The seeming simplicity of this final solution belies the significant computation effort required to search such a large decision space. For, as it shall be discussed in greater detail in subsequent sections, the runtime of the MOGADOR algorithm for the Fresno -- Tulare case study region was far and away the longest among all of the case study sites. 
    
        \begin{figure}[!h]
            \begin{center}
            \includegraphics[width=6in]{figures/Fresno_PathwayResults.png}   
            \caption{Fresno Region Corridor Analysis Results}
            \label{fig:Fresults}
            \end{center}
        \end{figure}

        \begin{figure}[!h]
            \begin{center}
            \includegraphics[width=5.5in]{figures/Fresno_PathwayLarge.png}   
            \caption{Fresno Region Top 100 Corridors Basin Wide Overview}
            \label{fig:FsolutionOverview}
            \end{center}
        \end{figure}

Figure \ref{fig:FelevationProfile} plots the along corridor elevation profile for the best output corridor solution generated by the MOGADOR algorithm with an initial seed population of 100,000. As the profile plot reveals, for the Fresno -- Tulare region, the corridor solution actually routes along a path of decreasing elevation from the source to the destination. This feature is unique among all of the case study regions, with all of the four others involving a route that moves progressively uphill to a higher elevation location. 
        
        \begin{figure}[!h]
            \begin{center}
            \includegraphics[width=5.5in]{figures/Fresno_Elevation_Profile.png}   
            \caption{Fresno Region Proposed Corridor Elevation Profile}
            \label{fig:FelevationProfile}
            \end{center}
        \end{figure}

\clearpage
    
\section{Evaluating the MOGADOR Algorithm's Runtime Performance}

Figure \ref{fig:Runtimes} depicts the runtime performance of the MOGADOR algorithm for the five case study regions previously introduced with treatments per case study. As the figure shows, the performance of the algorithm scales roughly linearly with respect to problem size -- as measured both in terms of the scale of the search domain and the breadth of the search effort. Due to the stochastic nature of the MOGADOR optimization routine and discrepancies between the source to destination separation distances as well as the morphological structure of the search domains for each of the case study regions, a more formal analysis of the runtime performance is discouraged so as not to generate misleading conclusions. 

    \begin{figure}[!h]
        \begin{center}
        \includegraphics[width=5.5in]{figures/Runtimes.png}
        \caption{Algorithm Runtime Performance for Each of the Five Case Study Regions for Three Population Sizes}
        \label{fig:Runtimes}
        \end{center}
    \end{figure}
    
Figure \ref{fig:Evolutions} plots the number of evolutionary iterations required to achieve convergence for each run of the MOGADOR algorithm across the five different case study regions and the three population sizes evaluated per region. No trend was expected between either the identity of the case study region or the population size. And, as the figure illustrates, indeed no trend was observed. 
    
    \begin{figure}[!h]
        \begin{center}
        \includegraphics[width=5.5in]{figures/Evolutions.png}
        \caption{Algorithm Convergence Rates for Each of the Five Case Study Regions for Three Population Sizes}
        \label{fig:Evolutions}
        \end{center}
    \end{figure}

\section{Evaluating the MOGADOR Algorithm's Solution Quality}

Figure \ref{fig:MarginImprovement} presents an analysis of the quality of the best output solutions generated by the MOGADOR algorithm for each of the case study regions compared relative to that of the euclidean shortest corridor linking the source location to the destination location. Two plot series are shown, the first, in red, described as the \textit{Margin of Deviation}, is the percent increase in the along path distance of the output solution corridor relative to that of the Euclidean shortest corridor. The expectation for this series is that all the values greater than or equal to 0\%. This expectation reflects the understanding that the optimal corridor solution must be at least as long as the Euclidean shortest corridor. As the Margin of Deviation plot series illustrates, the output solutions generated by the MOGADOR algorithm are 18\% to 22\% longer than the Euclidean shortest path alternatives.

The second plot series, in blue, depicts what is termed as the \textit{Margin of Improvement}. These values correspond to the percent decrease in the cumulative aggregate objective scores associated with the MOGADOR output corridor solution relative to that of the Euclidean shortest corridor. The initial expectation here is that these values should always be negative, or at least equal to zero, reflecting the degree to which the MOGADOR algorithm was able to deliver a solution that is improved, in terms of reduced along path cost, over the Euclidean shortest corridor. As the Margin of Improvement plot series illustrates, for Fresno -- Tulare, Oxnard, and Santa Barbara, the MOGADOR algorithm's best solutions achieved between a 37\% to 46\% reduction in cumulative aggregate objective scores over the Euclidean shortest corridor alternative. For the San-Bernadino and San-Diego regions the Margin of Improvement appears to be \%100. This feature reflects the fact that the Euclidean shortest corridor for these two regions exits the search domain and thus results in an arbitrarily high cumulative aggregate objective score value.

    \begin{figure}[!h]
        \begin{center}
        \includegraphics[width=5.5in]{figures/Margin_Improvement.png}
        \caption{Comparison of the Along Path Distance and Cumulative Objective Scores between the Solution Corridors and the Euclidean Shortest Corridors}
        \label{fig:MarginImprovement}
        \end{center}
    \end{figure}
    
\section{Evaluating the MOGADOR Algorithm's Solution Elevation Profiles}

Figure \ref{fig:ElevationProfiles} plots the along path elevation profiles for each of the best MOGADOR algorithm output corridor solutions relative to one another. This figure is useful for gauging (1) the relative length of each corridor (2) the relative elevation gain of each corridor and (3) the degree to which each corridor is trading a smooth accumulation of slope for routing either away from high intensity landuse areas or towards highly accessible areas. 

    \begin{figure}[!h]
        \begin{center}
        \includegraphics[width=5.5in]{figures/ElevationProfiles.png}
        \caption{Comparison of the Along Corridor Elevation Profiles for each of the Solution Corridors}
        \label{fig:ElevationProfiles}
        \end{center}
    \end{figure}
    
\section{Evaluating the Life-Cycle Water-Energy Usage Efficiencies of Proposed Reuse Systems}
    
Figure \ref{fig:Efficiencies} provides an information rich perspective on the final output results generated from the synthetic combination all three of the separate modeling components described in the previous chapters of the dissertation. It depicts, the ratio of water withdrawals and water consumption relative to the volume of water recovered by reuse. This ratio assumes that the rate of reuse equals 100\% of the permitted wastewater flows for each of the WWTPs in the five case study regions previously discussed. 

The water withdrawal and consumption figures presented are generated by combining a calculated instantaneous flow rate with the the proposed proposed corridor specification for each case study region into an expected annual pump energy consumption figure. This pump energy consumption figure is then translated into an expected water withdrawal figure through an interpretation of the fractional energy generation technology mix responsible for the production of electricity in each region. 

The horizontal red line plotted on the figure shows the critical threshold at which the energy consumed in the operation of a reuse systems results in either the withdrawal or consumption of more water -- at the point of electricity production -- than is able to be reclaimed by the reuse process. The range of values depicted by the plot series bands reflects an assumed range of pump operational efficiencies of between 25\% - 75\%.

In only one of the case study regions, Fresno -- Tulare, does there appear to be a net savings of water associated with the practice of reuse. This is largely due to the minimal pump energy requirements associated with the proposed reuse system in this region, which involves a corridor specification that routes downhill and is able to take advantage of gravity to overcome most of the total head associated with the water delivery effort. In all of the other case study regions, the range of consumption and withdrawals exceeds the critical threshold. This can be interpreted a situation where the practice of reuse is tantamount to the importation of water -- in the form of energy -- from the region in which the requisite energy has been produced.

These findings suggest that the systemic benefits of the reuse of treated wastewater for the practice of artificial groundwater reuse are not so great as initially assumed to be. Furthermore, they demonstrated that the efficiency of a reuse project is highly dependent upon the local geographic context in which the system is to be  implemented. 
    
    \begin{figure}[!h]
        \begin{center}
        \includegraphics[width=5.5in]{figures/Efficiencies.png}
        \caption{Comparison of the Net Water Usage Efficiencies of Reuse for each of the Case Study Regions Measured in Terms of Both the Withdrawals and Consumption of Water for the Production of Energy Required for Reuse}
        \label{fig:Efficiencies}
        \end{center}
    \end{figure}