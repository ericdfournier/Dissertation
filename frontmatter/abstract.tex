% the abstract

This dissertation investigates the dynamic energy-water usage efficiencies of civil engineering projects involving the recharge of subsurface groundwater aquifers via the reuse of treated municipal wastewater. To this end, a three-component model has been developed. The first component uses a cartographic modeling technique known as Weighted Overlay Analysis (WOA) to develop a quantitative understanding of the location and extent of geographic areas that are suitable as sites for groundwater recharge in a given geographic context. The second component uses a Genetic Algorithm (GA) to address the multi-objective spatial optimization problem associated with locating corridors for the support infrastructure required to physically transport water from the treatment facility to the recharge site. The third and final component takes data about the anticipated recharge treatment source location, reuse destination location, and proposed infrastructure corridor location and uses them to populate a spatially explicit Life Cycle Inventory (LCI) model capturing all of the system's process energy consumption and material inputs. Five case studies involving the planning of new basin scale artificial recharge systems within the state of California are discussed.